\documentclass[10pt,a4paper,titlepage]{article}
\usepackage[utf8]{inputenc}
\usepackage{amsmath}
\usepackage{amsfonts}
\usepackage{amssymb}
\usepackage{makeidx}
\usepackage{enumitem}
\usepackage{graphicx}
\usepackage{longtable}
\usepackage[hidelinks]{hyperref}

%this is a command used in the title template
\newcommand{\HRule}{\rule{\linewidth}{0.5mm}}

%questo fa in modo che le liste numerate siano allineate come le altre
\setenumerate{leftmargin=*, labelindent=\parindent}

%questo genera il toc, ricorda di eseguire due volte
\makeindex

\begin{document}
\begin{titlepage}
\begin{center}

%logo
\includegraphics[width=0.30\textwidth]{./images/logo}~\\[1cm]
\textsc{\LARGE Politecnico di Milano}\\[1.5cm]

\textsc{\Large Software Engineering 2 Project}\\[0.5cm]

% Title
\HRule \\[0.4cm]
{ \Huge \bfseries MeteoCal \\[0.4cm] }
{ \huge \bfseries Test Cases \\[0.4cm] }
\HRule \\[1.5cm]

% Author
\begin{flushright}
\noindent
\large
\emph{Authors:}\\
Andrea \textsc{Celli}\\
Stefano \textsc{Cereda}
\end{flushright}
\vfill

% Bottom of the page
{\large \today}

\end{center}
\end{titlepage}

\tableofcontents

\pagebreak
\part{Introduction}
In order to test our platform we provide the following documents:
\begin{itemize}
\item Requirements and analysis specification document (RASD)
\item Design Document (DD)
\item Source Code
\end{itemize}

\pagebreak
\part{Test cases}
The following test cases address the key functionalities of the MeteoCal platform.

\section{User registration}
\begin{tabular}{| p{0.2\linewidth} | p{0.8\linewidth} |}
\hline Goal & User registration\\
\hline Environment & Registration page\\
\hline Input & Random profile information \\
\hline Expected output & The registration is completed \\
\hline Obtained output & The same as above. The new user information are stored in the database \\
\hline Final output & The user is taken back to the login page where he/she has to perform the login \\
\hline Possible errors &
\begin{itemize}
\item  Empty fields: the system prevents the user registration to be
accomplished as all the fields are mandatory (previous inputs are not
deleted)
\item Not-valid email address: the system prevents the user registration to be
accomplished (previous inputs are not deleted)
\item Mismatching password: the system prevents the user registration to be
accomplished (previous inputs are not deleted)
\item Email already in use: the system prevents the user registration to be
accomplished (previous inputs are not deleted)
\end{itemize}\\
\hline
\end{tabular}


\section{Login}
\begin{tabular}{| p{0.2\linewidth} | p{0.8\linewidth} |}
\hline Goal & Login \\
\hline Environment & The login page \\
\hline Input & Email and password belonging to a registered user \\
\hline Expected output & The systems confirms the login \\
\hline Obtained output & The systems confirms the login \\
\hline Final output & The user is redirected to his homepage \\
\hline Possible errors &
\begin{itemize}
\item Empty fields: the systems doesn’t allow the login
\item Invalid email: the system doesn’t allow the login
\item Wrong password: the system doesn’t allow the login
\end{itemize}\\
\hline
\end{tabular}


\section{Create a new event}
\begin{tabular}{| p{0.2\linewidth} | p{0.8\linewidth} |}
\hline Goal & Create a new event \\
\hline Environment & Homepage. More specifically the user will use the dialog appeared
(after the “New Event” button is pressed) on the right of the calendar \\
\hline Input & The relevant event details \\
\hline Expected output & The system confirms the creation of the event \\
\hline Obtained output & The system stores the event details in the database and, if
necessary, generates the notifications related to the new event \\
\hline Final output & The new event is displayed in the user’s calendar and the panel on
the right is hidden \\
\hline Possible errors &
\begin{itemize}
\item The place is not selected among the available choices: the event is
created without a location
\item The user is not selected among the available choices: the system
prevents the creation of the event and reset previously specified data.
\item (Time consistency of the start/end date is always guaranteed)
\end{itemize}\\
\hline
\end{tabular}


\section{Delete event}
\begin{tabular}{| p{0.2\linewidth} | p{0.8\linewidth} |}
\hline Goal & Delete event \\
\hline Environment & Homepage, the dialog showing the event information \\
\hline Input & Delete button \\
\hline Expected output & The system confirms the deletion of the event \\
\hline Obtained output & The systems deletes the event from the Event table. It also
deletes all the notifications and participations related to that event \\
\hline Final output & The event is no longer displayed in the calendar \\
\hline Possible errors & none \\
\hline
\end{tabular}


\section{Remove participation}
\begin{tabular}{| p{0.2\linewidth} | p{0.8\linewidth} |}
\hline Goal & Remove participation \\
\hline Environment & Homepage, the dialogs showing event information \\
\hline Input & The Remove Participation button \\
\hline Expected output & The systems confirms the unregister request \\
\hline Obtained output & The user participations is removed from the database. All the
notifications for the user related to that event are deleted \\
\hline Final output & The events and the related notifications are deleted from the calendar \\
\hline Possible errors & none \\
\hline
\end{tabular}


\section{Modify event}
\begin{tabular}{| p{0.2\linewidth} | p{0.8\linewidth} |}
\hline Goal & Modify event \\
\hline Environment & Homepage \\
\hline Input & Modify button and then any details that the user may want to change \\
\hline Expected output & The system allows the user to modify the event details \\
\hline Obtained output & The system updates the event in the database. It also produces
event changed notifications for participants, if any \\
\hline Final output & The updated event is displayed in the calendar \\
\hline Possible errors & The same of “create of an event” \\
\hline
\end{tabular}


\section{Read and answer to notifications}
\begin{tabular}{| p{0.2\linewidth} | p{0.8\linewidth} |}
\hline Goal & Read and answer to notifications \\
\hline Environment & The homepage \\
\hline Input & One of the possible answer buttons for notifications \\
\hline Expected output & The systems confirms the user’s answer to the notification \\
\hline Obtained output & The systems change the state of the notification in the database
and update its answer (if the notification type requires one) \\
\hline Final output & The notification is removed from the panel on the left and, eventually,
the related event is updated in the calendar. \\
\hline Possible errors & none\\
\hline
\end{tabular}


\section{Search User}
\begin{tabular}{| p{0.2\linewidth} | p{0.8\linewidth} |}
\hline Goal & Search User \\
\hline Environment & Any page of the system available to a logged user \\
\hline Input & The search key \\
\hline Expected output & The system performs the search \\
\hline Obtained output & The systems retrieves the search results from the database \\
\hline Final output & Search results (if any) are displayed in the result page \\
\hline Possible errors & Not valid search key: the search key can be one between the username,
the name, or the surname of the user. If the system does not find any
matching with its users in one of those fields it will show no results \\
\hline
\end{tabular}


\section{Selection of the user}
\begin{tabular}{| p{0.2\linewidth} | p{0.8\linewidth} |}
\hline Goal & Selection of the user\\
\hline Environment & Result page \\
\hline Input & The selected user \\
\hline Expected output & The system confirms the selection \\
\hline Obtained output & The system retrieves selected user data from the db \\
\hline Final output & The selected user calendar is displayed \\
\hline Possible errors & none \\
\hline
\end{tabular}


\section{Browse other user's calendar}
\begin{tabular}{| p{0.2\linewidth} | p{0.8\linewidth} |}
\hline Goal & Browse other user’s calendar\\
\hline Environment & external calendar page \\
\hline Input & The selection of an event \\
\hline Expected output & The system confirms the selection \\
\hline Obtained output & The system retrieves event’s details \\
\hline Final output & The system shows the details allowed by the event’s privacy setting \\
\hline Possible errors & none \\
\hline
\end{tabular}


\section{Change settings}
\begin{tabular}{| p{0.2\linewidth} | p{0.8\linewidth} |}
\hline Goal & Change settings \\
\hline Environment & Settings page \\
\hline Input & New user information \\
\hline Expected output & The system confirms the new user information \\
\hline Obtained output & The system updates the user data stored in the database \\
\hline Final output & The user is redirected to his homepage \\
\hline Possible errors &
\begin{itemize}
\item Empty fields: the system prevents the data update accomplishment
(previous inputs are not deleted).
\item Mismatch between passwords: the system prevents the data update
accomplishment (previous inputs are not deleted).
\end{itemize}\\
\hline
\end{tabular}


\pagebreak
\part{Note on the automated tests}
The automated tests can perform a huge number of queries to the OpenWeatherMap service, therefore it is possible that sometimes they occasionally fail (OpenWeatherMap grants a limited number of query per minute). This is not a problem as the tests will start again to work of you wait some time.
 
\end{document}